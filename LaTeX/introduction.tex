\section{Introduction}

\drawBorder

Gravitational two body problem is a classical mechanics problem of predicting the motion of two massive objects (abstractly viewed as point particles) under gravity. We make use of Newton's law of gravity and numerical methods to find the solution for the orbits of these objects, and simulate their changing positions.

\vspace{1cm}

Let $m_1$ and $m_2$ be the masses of two massive objects assumed to be point particles, separated by a distance $r$. We assume that the gravitational interaction is only with these two bodies, and there are no bodies nearby other than the two under consideration.

From Newton's law of gravity, the force $F$ experienced by an object of mass $m$ due to another object of mass $M$ separated by a distance of $R$ is given by,

\begin{equation}
    F = \frac{G M m}{R^2}
\end{equation}

Where $G$ is the Universal gravitational constant, having the value of
$\SI{6.674e-11}{\newton \metre^2 \per \kg ^2}$\\

In vector form, this equation will be

\begin{equation}
    \vec{F} = \frac{G M m}{\vec{\abs{R}}^{\, 2}} \hat{R}
\end{equation}

Where \textbf{R} is the radius vector from mass $M$ to mass $m$.\\

Therefore, the gravitational force between masses $m_1$ and $m_2$ is

\begin{equation}\label{force_eqn}
    \begin{split}
        F &= \frac{G m_1 m_2}{r^2}\\
        \vec{F} &= \frac{G m_1 m_2}{\vec{\abs{r_{12}}}^{\, 2}} \hat{r_{12}}
    \end{split}
\end{equation}

where $\vec{r_{12}}$ is the vector from $m_1$ to $m_2$, and $\hat{r_{12}}$ is the unit vector in it's direction.\\

Using Newton's second law of motion, $\vec F=m \vec a$, we can calculate the acceleration of a body.\\

Let $\vec F_{12}$ be the force on mass $m_1$ by $m_2$, then the acceleration of mass $m_1$ will be,

\begin{equation}
    \begin{split}
        \vec F_{12} &= m_{1}\vec a_1\\
        \vec a_1 &= \frac{\vec F_{12}}{m_1}\\
        \frac{dv_1}{dt} ={} &{} \vec{\dot{v_1}} = \frac{\vec F_{12}}{m_1}\\
        \frac{d^2x_1}{dt} ={} &{} \vec{\ddot {x_1}} = \frac{\vec F_{12}}{m_1} \label{displacement_eqn}
    \end{split}
\end{equation}

Integrating equation (\ref{displacement_eqn}) twice using suitable numerical integration method, we can determine the position $x(t)$ of mass $m_1$. Similar calculation can be done to find $x(t)$ of $m_2$.\\

We make use of \textbf{algorithm 2.1} mentioned in \textcite{orbital_mechanics_4ed} to compute the motion of two bodies in an inertial frame of reference.

The python code for the simulation makes use of \textbf{Numpy}, \textbf{Scipy} \& \textbf{Matplotlib} libraries. There are 2 versions of the code, one where the motion of the bodies are with respect to a non-inertial frame of reference, and the other where the motion is relative to one of the body.
